%D \module
%D   [       file=t-gnuplot,
%D        version=2007.02.07,
%D          title=\CONTEXT\ Extra Modules,
%D       subtitle=\GNUPLOT\ Inclusion,
%D         author={Hans Hagen, Taco Hoekwater, Mojca Miklavec, (Aditya Mahajan)},
%D           date=\currentdate,
%D      copyright=\PRAGMA]
%C
%C This module is part of the \CONTEXT\ macro||package and is
%C therefore copyrighted by \PRAGMA. See mreadme.pdf for
%C details.

%D This module is used for creating gnuplot graphs on-the-fly and including them
%D into documents.
%D
%D Known Bugs:
%D
%D \startitemize
%D \item spurious space \& page
%D \stopitemize
%D 
%D TODO (Optimisations):
%D
%D \startitemize
%D \item optimize the number of gnuplot runs (if possible, gnuplot should be run only once)
%D \item optimize the number of times for loading/converting an already used graphic
%D \item pstopdf is a bit slow
%D \stopitemize
%D
%D TODO (Handle things that may go wrong):
%D - gnuplot executable doesn't exist
%D - context terminal isn't available or some other failure in gnuplot script (no file created)
%D - write18 disabled (you may call gnuplot later - create a script)
%D
%D TODO (Missing functionality):
%D
%D - a lot ...

\writestatus{loading}{Gnuplot module}

\startmodule[gnuplot]

\unprotect

\def\c!terminal {terminal}
\def\c!options  {options}
\def\c!pointset {pointset}
\def\c!purge    {purge}

%D MPextensions
%D
%D XXX
%D if possible, they should be specific to \type{\startGNUPLOTgraphic},
%D so unvisible to \type{MPcode}
%D (probably something like \type+\appendtoks ... to\everyGNUPLOTgraphic+)
\startMPextensions
	% load metapost macros (only once)
	input mp-gnuplot.mp ;
	% number of points defined with \setupGNUPLOTterminal[pointset=...]
	%gp_num_points_with_tex := \gp:num:pointswithtex;
	gp_num_points_with_tex := 3;
	% main color should be set equal to the current text color
	gp_color_foreground := \MPcolor{currentcolor};
	gp_color_lt[-2] := gp_color_foreground;
	% TODO: is there any chance to make this local to gnuplot?
	% linejoin & linecap
	linejoin := \@@GNUPLOT@term@context@linejoin;
%	linecap  := \gp:term:context:linecap;
	% dashes or solid? (true/false)
	gp_use_dashed := \@@GNUPLOT@term@context@is@dashed;
	% dashlength scale
%	gp_scale_dashlength := \gp:term:context:dashlength;
	% linewidth scale
	gp_scale_linewidth := \@@GNUPLOT@term@context@linewidth;
	
	gp_scale_text := \@@GNUPLOT@term@context@textscale;

	% linejoin := \gp:term:context:linejoin;
	% linecap  := \gp:term:context:linecap;
	% % dashes or solid? (true/false)
	% gp_use_dashed := \gp:term:context:dashed;
	% % dashlength scale
	% gp_scale_dashlength := \gp:term:context:dashlength;
	% % linewidth scale
	% gp_scale_linewidth := \gp:term:context:linewidth;
	
	gp_points_with := gp_points_with_\@@GNUPLOT@term@context@points;
	
\stopMPextensions

%D (hopefully) temporary solution for handling the color in expressions like
%D    \type+draw \sometxt{...} withcolor red+
\chardef\TeXtextcolormode\zerocount

%D We have to make sure that runtime MP graphics are enabled,
%D otherwise sizes of labels created by \type+\sometxt{}+ would be wrong
%D if user didn't enable that option in \type{cont-usr.tex}.
%D (Taco says it might be a bug, but let's not worry too much about it.
%D  If it will be resolved, we may delete this.)
\runMPgraphicstrue

%D We need a \type{\strut} in front of labels for better vertical centering.
%D This might still fail for Zapfino and alike where \type{\strut} might be smaller
%D than the actual font height.
%D
%D TODO (optional improvement): instead of placing \type{\strut} in front,
%D create a \type{\hbox} and adjust it's dimensions to \type{\strut}'s height and depth.

%D Aditya, thanks a lot for requesting it!
%D Hans, thanks a lot for implementing this!
\definetextext[gp]{\strut}
%D TODO (feature request):
%D \starttyping
%D    \sometxt[gp][ss,20pt]{abc}
%D \stoptyping
%D should become equivalent to
%D \starttyping
%D    \sometxt{\switchtobodyfont[ss,20pt]\strut abc}
%D \stoptyping
%D
%D Or, even more drastic perhaps, I would love to implement
%D \starttyping
%D    \sometxt[gp][iwona,bold,c]{abc}
%D \stoptyping
%D meaning: iwona, bold typeface, size \quote{c}

%%%%%%%%%%%%%%%%%%%%%%%%%%%%%%%%%%%%%%%%%%%%%%%%%%%%%%%%%%

%D The following patch has been written by Aditya and seems to work OK,
%D however it would be more clean to have this functionality in the core.
%D Redefining low-level macros might be a bit dangerous, esp. since
%D they tend to change over time.
%D
%D At the moment the hack only works in MKII; for MKIV it needs to be fixed
%D
%D TODO: remind Hans to implement it in core ;)
%D       and remove this patch from the module
%D
\ifx\directlua\undefined
\long\def\redofiltersometxt[#1]%
{\doifnextcharelse[{\reredofiltersometxt[#1]}{\redodofiltersometxt[#1]}}

\long\def\redodofiltersometxt[#1]#2%
  {\increment\txtcounter
   \TeXtext[#1]\txtcounter{#2}%
   \filtersometxt}

\long\def\reredofiltersometxt[#1][#2]#3%
  {\increment\txtcounter
   \TeXtext[#1]\txtcounter{\switchtobodyfont[#2]\strut#3}%
   \filtersometxt}
\fi

%%%%%%%%%%%%%%%%%%%%%%%%%%%%%%%%%%%%%%%%%%%%%%%%%%%%%%%%%%


%D This was needed for the old method of
%D \type{put_text("some text")} using textext: not used/supported any more
%\ifx\directlua\undefined
%	\forceMPTEXcheck{put_text}
%\fi
%\runMPTEXgraphicstrue

\newcounter\GNUPLOTnumber

%D \macros
%D   {startGNUPLOTinclusions, resetGNUPLOTinclusions}
%D
%D For those who want to have two or more graphs with similar options,
%D these options may be included inside \type{\startMPinclusions ... \stopMPinclusions}
%D and will be place on the top of the created \GNUPLOT\ script.
%D
%D \starttyping
%D \startGNUPLOTinclusions
%D    set xlabel '$x$'
%D    set ylabel '$y$'
%D    set format y "%.1f"
%D \stopGNUPLOTinclusions
%D
%D \startGNUPLOTscript[sin]
%D    plot sin(x)
%D \stopGNUPLOTscript
%D \startGNUPLOTscript[cos]
%D    plot cos(x)
%D \stopGNUPLOTscript
%D \stoptyping
\long\def\startGNUPLOTinclusions
  {\def\stopGNUPLOTinclusions{\ifx\savebuffer\undefined \else \savebuffer[gnuplot-inclusions]\fi}%
   \dostartbuffer[gnuplot-inclusions][startGNUPLOTinclusions][stopGNUPLOTinclusions]}

% \def\resetGNUPLOTinclusions{\let\GNUPLOTinclusions\empty}

% creates an empty file (there must be a cleaner way to do it)
\def\resetGNUPLOTinclusions
  {\immediate\openout\scratchwrite=\jobname-gnuplot-inclusions.tmp
   \immediate\closeout\scratchwrite}

%\startbuffer[gnuplot-inclusions]\stopbuffer\ifx\savebuffer\undefined \else \savebuffer[gnuplot-inclusions]\fi}

\resetGNUPLOTinclusions

%D On the other hand, one can probably achieve the same effect
%D when drawing two plots inside the same script, like that:
%D
%D \starttyping
%D \startGNUPLOTscript[sin and cos]
%D    set xlabel '$x$'
%D    set ylabel '$y$'
%D    set format y "%.1f"
%D    plot sin(x)
%D    plot cos(x)
%D \stopGNUPLOTscript
%D \stoptyping
%D
%D and then recall the graphics using \type{\useGNUPLOTgraphic[sin and cos][1]}.

%D \macros
%D   {startGNUPLOTscript}
%D
%D \starttyping
%D \startGNUPLOTscript{some name}
%D   plot sin(x)
%D \stopGNUPLOTscript
%D \stoptyping

\def\startGNUPLOTscript
  {\bgroup\dosingleempty\dostartGNUPLOTscript}

% \def\redostartGNUPLOTscript#1
%   {\obeylines
%    \catcode`\%=\@@letter
%    \catcode`\|=\@@letter
%    \catcode`\$=\@@letter
%    \dodostartGNUPLOTscript{#1}%
% %\dostartGNUPLOTscript[#2]
% }
% 
% \def\dostartGNUPLOTscript[#1]%
%   {\iffirstargument
%      \obeylines
%      \catcode`\%=\@@letter
%      \catcode`\|=\@@letter
%      \catcode`\$=\@@letter
%      \dodostartGNUPLOTscript{#1}%
%    \else
%      \redostartGNUPLOTscript
%    \fi
% }

\def\dostartGNUPLOTscript[#1]%
  {%\iffirstargument
     \dodostartGNUPLOTscript{#1}%
   %\else
   %  \dodostartGNUPLOTscript
   %\fi
}

% gps:n:{name} = gnuplotscript : number : {name} - number of script with name {name}
\long\def\dodostartGNUPLOTscript#1%#2\stopGNUPLOTscript
  {\doglobal\increment\GNUPLOTnumber
   \letgvalue{gps:n:#1}\GNUPLOTnumber
   % in case of LuaTeX we need to write the buffer into file explicitely
   \def\stopGNUPLOTscript{\egroup \ifx\savebuffer\undefined \else \savebuffer[gnuplot-\GNUPLOTnumber]\fi}%
   \dostartbuffer[gnuplot-\GNUPLOTnumber][startGNUPLOTscript][stopGNUPLOTscript]%
   }

% When are the graphics processed/read?
%
% - \gps:n:{name} (gnuplot script:name:{name}) holds the number of gnuplot script;
%   - that number was defined if we created the plot using \startGNUPLOTscript{name}
%   - and most probably undefined if we only issued \processGNUPLOTfile[name][filename]
%     => if, at the time of issuing \processGNUPLOTfile[name][filename], \gps:n:{name} is not defined,
%        it should be defined at that time
% - \gpe:{name}:{terminal} (gnuplot graphic executed:{name}:{terminal})
%   is defined if we executed the command for that specific name and for that terminal
%   (once it will probably be 0 for a failed run and 1 for a successful one)

% \useGNUPLOTgraphic[name] has three different ways of working:
% - it can be called after \processGNUPLOTfile[name][filename]
%   which defined MP graphics that are now used
% - it can be called for the first time under the current terminal
%   in that case it compiles the graphic and includes it
% - it can be called for the second, third, ... time
%   in which case it only includes files without compiling it

%D \macros
%D   {useGNUPLOTgraphic}

% 5 ways of calling it:
% - \useGNUPLOTgraphic{name}
% - \useGNUPLOTgraphic[name]
% - \useGNUPLOTgraphic[name][1,2,5]
% - \useGNUPLOTgraphic[name][width=.8\textwidth]
% - \useGNUPLOTgraphic[name][1,2,5][width=.8\textwidth]

% this code takes care of reading arguments
\def\useGNUPLOTgraphic
	{\dotripleempty\douseGNUPLOTgraphic}

\def\douseGNUPLOTgraphic[#1][#2][#3]%
	{\doifelse{#3}{}{%
		% < 3 arguments
		\doifelse{#2}{}%
			% 1 argument
			% as in \useGNUPLOTgraphic[name]
			{\dodouseGNUPLOTgraphic[#1][][]}%
			% % as in \useGNUPLOTgraphic{name}
			% {\redouseGNUPLOTgraphic[][][]}%
			% 2 arguments
			{\doifassignmentelse{#2}%
				% as in \useGNUPLOTgraphic[name][width=.8\textwidth]
				{\dodouseGNUPLOTgraphic[#1][][#2]}%
				% as in \useGNUPLOTgraphic[name][1,2,5]
				{\dodouseGNUPLOTgraphic[#1][#2][]}%
			}%
		}%
		% as in \useGNUPLOTgraphic[name][1,2,5][width=.8\textwidth]
		{\dodouseGNUPLOTgraphic[#1][#2][#3]}%
	}

\def\redouseGNUPLOTgraphic[#1][#2][#3]#4%
	{\dodouseGNUPLOTgraphic[#4][][]}

% and this code actually does something with it

%D \doifundefined      {string}    {...}
%D \doifdefined        {string}    {...}
%D \doifundefinedelse  {string}    {then ...} {else ...}
%D \doifdefinedelse    {string}    {then ...} {else ...}

% \doifGNUPLOTscriptdefined{name}{...}

%  if gnuplot script with {name} and current terminal has already been processed
% \doifGNUPLOTscriptprocessed{name}{...}

% \letGNUPLOTscriptprocessed[optional terminal]{name} signals that the gnuplot script named {name}
% has already been processed with the current terminal
% TODO: currently it is always defined to be one as soon as one tries to process it,
%       even if en error is produced; in future it would be helpful if it would be set to zero
%       if it was unsuccessfully executed;
% that is needed, since scripts are processed only when one first asks for including the graphic
\def\letGNUPLOTscriptprocessed
	{\dosingleempty\doletGNUPLOTscriptprocessed}
%\def\doletGNUPLOTscriptprocessed[#1]#2%
%{\letgvalue{gpe:#1:\@@GNUPLOTterminal}\plusone}
\def\doletGNUPLOTscriptprocessed[#1]#2%
	{\iffirstargument
	 	% terminal has been specified
	 	\letgvalue{gpe:#2:#1}\plusone
	 \else
	 	% no terminal specified - use the current one
	 	\letgvalue{gpe:#2:\@@GNUPLOTterminal}\plusone
	 \fi}

% Although that should preferably not happen, one might come to an idea of defining
% a gnuplot script with the same name as already defined.
% In that case the old script cannot be referenced any more, but we can still try
% to do out best to make it work anyway. We have to do two things:
% - claim that script with that name hasn't been defined yet, so that processing will happen again
%   (TODO: do it in a more elegant way for all known terminals)
% - undefine any metapost graphics (TODO: I have no idea yet how it can be done)
%
% \resetGNUPLOTscriptprocessed{name}
\def\resetGNUPLOTscriptprocessed#1%
	{\bgroup
		% claim that the script with {name} (#1) hasn't been processed with {terminal} (##1) yet
		\def\undefineGNUPLOTscriptforterminal##1{\letbeundefined{gpe:#1:##1}}%
		% TODO: the list of available terminals should be generated automatically
		\processcommalist[context,postscript,ps,eps,pdf,metapost,mp,png]\undefineGNUPLOTscriptforterminal
		\resetGNUPLOTgraphics{#1}%
	\egroup}

% \resetGNUPLOTgraphics{name} undefines any gnuplot graphic defined with \startGNUPLOTgraphic[name][number]
\def\resetGNUPLOTgraphics#1%
	{\doloop
		{\doifMPgraphicelse{gpg:#1:\recurselevel}%
		% TODO: be aware - @@MPG is low-level ConTeXt variable which might change without notice,
		% it would be better to call this \undefineMPgraphic{gpg:#1:\recurselevel} or something similar
		{\letbeundefined{@@MPGgpg:#1:\recurselevel}}%
		{\exitloop}}%
	}%

% TODO: this can probably be done in a better way
\def\doifGNUPLOTscriptprocessed#1#2%
	{\doifdefined      {gpe:#1:\@@GNUPLOTterminal}{#2}}
\def\doifGNUPLOTscriptprocessedelse#1#2#3%
	{\doifdefinedelse  {gpe:#1:\@@GNUPLOTterminal}{#2}{#3}}
\def\doifGNUPLOTscriptnotprocessed#1#2%
	{\doifundefined    {gpe:#1:\@@GNUPLOTterminal}{#2}}
\def\doifGNUPLOTscriptnotprocessedelse#1#2#3%
	{\doifundefinedelse{gpe:#1:\@@GNUPLOTterminal}{#2}{#3}}


% private
% \writeandprocessGNUPLOTscript{name}
\def\writeandprocessGNUPLOTscript#1%
	% TODO: if gps:n:#1 (holding the script content) is not defined,
	% error or warning should be issued
	%
	% only process the script if it has been defined and not processed before for the current terminal
	{\doifdefined{gps:n:#1}{\doifGNUPLOTscriptnotprocessed{#1}{%
		% \gpe:{name}:{terminal} is defined
		%\setgvalue{gpe:#1:\@@GNUPLOTterminal}{#2}
		\letGNUPLOTscriptprocessed{#1}%
		% call to gnuplot and processing/converting the graphics is only needed in the first ConTeXt run
		\doifmode{*\v!first}{% (perhaps also: if files haven't changed)
			\bgroup
			% catcodes trickery & alike
			%\the\everyGNUPLOTscript
			% open file for writing
			\immediate\openout\scratchwrite=\GNUPLOTfile.plt
			% TODO: terminal-specific options (default or provided by the user)
			\immediate\write\scratchwrite{\letterhash\space Do not modify this file - all changes will be overwritten}%
			\immediate\write\scratchwrite{\letterhash\space Change \jobname.tex instead.}%
			\immediate\write\scratchwrite{set terminal \@@GNUPLOToutput\space\@@GNUPLOToptions}%
			% add common inclusions in scripts for multiple similar plots
			%\doifnotempty
			%	{\GNUPLOTinclusions}{\immediate\write\scratchwrite{\GNUPLOTinclusions}}%
			% include common gnuplot 'inclusions'
			\immediate\write\scratchwrite{load '\jobname-gnuplot-inclusions.tmp'}%
			% output file
			\immediate\write\scratchwrite{set output "\@@GNUPLOTresult"}%
			% write main contents of the script, like "plot sin(x)"
			%\immediate\write\scratchwrite{\getvalue{gps:d:\GNUPLOTnumber}}%
			\immediate\write\scratchwrite{load '\GNUPLOTfile.tmp'}%
			\immediate\closeout\scratchwrite
			\egroup
			% TODO:
			% - check the state of write18 and warn the user if it's disabled,
			%   otherwise just everyone will start complaining that the module doens't work
			% - check if execution was successful; possible pitfalls:
			%   - gnuplot doesn't exist as a binary
			%   - gnuplot doesn't support context terminal
			%     or some other error in script which results in empty output file
			%
			% run gnuplot & execute the script that has just been written
			% \executesystemcommand{texmfstart --ifchanged=\GNUPLOTfile.plt\space
			% 	--direct \@@GNUPLOTprogram\space\GNUPLOTfile.plt}%
			\executesystemcommand{texmfstart --direct \@@GNUPLOTprogram\space\GNUPLOTfile.plt}%
			\writestatus{aaa}{texmfstart --direct \@@GNUPLOTprogram\space\GNUPLOTfile.plt}%
			\convertGNUPLOTgraphic
		}%
		% for ConTeXt terminal only - read the result
		\doif{\@@GNUPLOTterminal}{context}{\processGNUPLOTfile[#1][\@@GNUPLOTresult]}%
	}}}

% \dodouseGNUPLOTgraphic[name][numbers][options]
\def\dodouseGNUPLOTgraphic[#1][#2][#3]%
	{\bgroup
	\doifdefined{gps:n:#1}
		{\edef\GNUPLOTnumber{\getvalue{gps:n:#1}}%
		 \edef\GNUPLOTfile  {\bufferprefix gnuplot-\GNUPLOTnumber}%
		\writeandprocessGNUPLOTscript{#1}%
		% "ctxtools --purge" should delete the gnuplot script and other intermediate files
		% (but it seems that they are deleted automatically already; preferred or not?)
%		\registertempfile{\GNUPLOTfile.plt}
%		\registertempfile{\@@GNUPLOTresult}
%		\registertempfile{\@@GNUPLOTfinalresult}
		\doifelse{\@@GNUPLOTterminal}{context}%
		% for ConTeXt terminal only
		{\doifelse{#2}{}%
			% if no explicit figure number was specified, include all the figures
			{\doloop
				{\doifMPgraphicelse{gpg:#1:\recurselevel}%
				{\scale[#3]{\reuseMPgraphic{gpg:#1:\recurselevel}}}%
				{\exitloop}}}%
			% if numbers were specified, include the figures specified in the list only
			{\begingroup
				\def\useGNUPLOTgraphicN##1{\doifMPgraphicelse
					{gpg:#1:##1}%
					{\scale[#3]{\reuseMPgraphic{gpg:#1:##1}}}%
					% if graphic doesn't exist: draw a dummy frame instead and warn the user
					{\scale[#3]{\framed[frame=on,width=5in,height=3in,align={middle,lohi}]{GNUPLOT graphic #1, Nr. ##1 doesn't exist}}}}%
				\processcommalist[#2]\useGNUPLOTgraphicN
				\endgroup
			}%
		}%
		% for all the other terminals
		% TODO: add more safety switches (if pages don't exist for example)
		{\doifelse{#2}{}%
			% if no explicite figure number was specified, include all the pages
			% TODO: properly handle METAPOST & PNG (only works for (E)PS & PDF) !!!
			{\getfiguredimensions[\@@GNUPLOTresult]%
				\dorecurse{\noffigurepages}{\externalfigure[\@@GNUPLOTfinalresult][page=\recurselevel,#3]}}%
			% if numbers were specified, include the pages specified in the list only
			{\begingroup
				\def\useGNUPLOTgraphicN##1{\externalfigure[\@@GNUPLOTfinalresult][page=##1,#3]}%
				\processcommalist[#2]\useGNUPLOTgraphicN
				\endgroup
			}%
		}%
	}%
	\egroup}

%D \macros
%D   {setupGNUPLOT}

\def\setupGNUPLOT
%  {\dodoubleempty\getparameters[@@GNUPLOT]}
  {\dosingleargument\dosetupGNUPLOT}

\def\dosetupGNUPLOT[#1]%
  {\getparameters[@@GNUPLOT][#1]%
   % define all the necessary points according to the option "pointset"
   %
   % XXX: no idea why this is needed, but otherwise it complains that @@GNUPLOTpointset is undefined
   \edef\currentGNUPLOTpointset{\@@GNUPLOTpointset}%
   \ifx\directlua\undefined
   \startTeXtexts
    \doloop{\doifelseconversionnumber{\@@GNUPLOTpointset}{\recurselevel}%
       % +500 is a hack (hopefully the plot doesn't contain more than 500 labels)
       % otherwise the points would be overwritten by labels with another \TeXtext:
       % it might need a fix in ConTeXt core
       {\TeXtext{\numexpr\recurselevel+500\relax}{{\convertnumber{\currentGNUPLOTpointset}{\recurselevel}}}}%
       {\exitloop}}%
   \stopTeXtexts
   \fi
   % \gp:num:pointswithtex is passed to metapost, so that it knows
   % how many points are defined and chooses the proper point
   % form a set of the defined ones
   %
   % for safety reasons define \gp:num:pointswithtex to be equal to 1 (it can only increase later), otherwise:
   % - (something mod 0) won't work
   % - if conversion is not defined, the number will retain its old value (not desirable)
   \edef\gp:num:pointswithtex{1}%
   \doloop{\doifelseconversionnumber{\@@GNUPLOTpointset}{\recurselevel}%
       {\edef\gp:num:pointswithtex{\recurselevel}}%
       {\exitloop}}%
   % TODO:
   % - issue a warning if user wants to use points with TeX,
   %   but pointset= is undefined (if the first point doesn't exist)
   % - no idea what happens if conversion is infinite,
   %   so try to stop at some reasonable value (100?)
   % - current implementation redefines the points even if only terminal type
   %   has been set to some other value (which is a stupid approach, but I
   %   wanted to have a working version first and will consider efficiency later)
   % - it may be that the old points remain defined if conversion changes
   %   (perhaps they should be undefined again?)
  }

% Hans has written this piece of code, but:
% - "start" and "/MIN" caused problems
% - pgnuplot is not much more "portable" than gnuplot
%   best thing to do is to create a "gnuplot.bat" somewhere in PATH
%
%\def\processGNUPLOTscript
%  {\doifelse\operatingsystem{mswin}
%     {\executesystemcommand{start /MIN pgnuplot \GNUPLOTfile.plt}} % start is needed else gp fails
%     {\executesystemcommand{gnuplot \GNUPLOTfile.plt}}}
%\def\processGNUPLOTscript
%  {\executesystemcommand{gnuplot \GNUPLOTfile.plt}}

% TODO: check if write18 is enabled; if not, issue a command and warn the user that running the module might be pointless or that he/she has to run gnuplot on the produced files manually
\def\convertGNUPLOTgraphic
  {\doifsomething\@@GNUPLOThandle{\executesystemcommand{\@@GNUPLOThandle}}}

%D \macros
%D   {processGNUPLOTfile}

%D It's needed to input a file resulting from a gnuplot run (with ConTeXt terminal).
%D It reads the file and "saves" the metapost graphics defined in that file,
%D so that they can be used with \usegnuplotgraphic (low level: \reuseMPgraphic) later
%D
% \processGNUPLOTfile[NAME][filename]
\def\processGNUPLOTfile
  {\dodoubleargument\doprocessGNUPLOTfile}

\def\doprocessGNUPLOTfile[#1][#2]%
  {% we first define two commands: \startGNUPLOTgraphic & \stopGNUPLOTgraphic to read the files in;
   % files that gnuplot creates in non-standalone mode look approximately like this:
   %
   % \startGNUPLOTgraphic[1]
   % ... metapost commands to draw the graph ...
   % \stopGNUPLOTgraphic
   % \startGNUPLOTgraphic[2]
   % ... metapost commands to draw the graph ...
   % \stopGNUPLOTgraphic
   % ...
   %
   % while reading the file in, metapost graphics named "gpg:{name}:{number}" are defined
   %
   % (we might need some additional arguments later on, but for now the figure number should suffice)
   \def\startGNUPLOTgraphic
      {%\obeyMPlines % <- no longer a problem
       \def\obeyedline{}% <- thanks to this
       \dosingleargument\dostartGNUPLOTgraphic}%
   \long\def\dostartGNUPLOTgraphic[##1]##2\stopGNUPLOTgraphic
      {\startreusableMPgraphic{gpg:#1:##1}##2\stopreusableMPgraphic}%
   %
   % if gps:n:{name} (which is a number) is not defined,
   % then we first have to increase the \GNUPLOTnumber counter and associate this name with the new number

   % , then we should now also define
   % gpe:{number}:context which should signal later on that the graphic
%   \doifdefined{gps:n:#1}
   % input the file
   % TODO: error / warning if the files doesn't exist
   \readlocfile{#2}{}{}% the third argument should be: if file not found
   % this file is known
   % TODO \letgvalue{gpf:#1}\plusone
  }

%D \macros
%D   {startGNUPLOTpage}
%D
%D Used for standalone \GNUPLOT\ figures & written out by gnuplot in standalone mode.
%D
%D Instead of having to \type{\useGNUPLOTgraphic}, a figure is inserted directly using
%D \type{\startMPpage ... }\type{\stopMPpage}.
%D
%D A high-level command is provided if some more advance features will
%D be needed in the future and to be able to ensure backward compatibility.

%D We need to preserve line breaks, otherwise metapost runs into troubles for longer input.
\def\startGNUPLOTpage
	{\begingroup\obeyMPlines\dostartGNUPLOTpage}

\long\def\dostartGNUPLOTpage#1\stopGNUPLOTpage
	{\endgroup\startMPpage#1\stopMPpage}


% TODO: currently, the following definitions are used,
%       but the ones below would be preferred

\def\defineGNUPLOThandle#1#2#3#4#5#6% name; output; suffix; conversion-method; gnuplot's result; final result
  {\setvalue{@@GNUPLOT-#1}{{#2}{#3}{#4}{#5}{#6}}}

\def\@@GNUPLOToutput{\filterfromvalue{@@GNUPLOT-\@@GNUPLOTterminal}51}
\def\@@GNUPLOTsuffix{\filterfromvalue{@@GNUPLOT-\@@GNUPLOTterminal}52}
\def\@@GNUPLOThandle{\filterfromvalue{@@GNUPLOT-\@@GNUPLOTterminal}53}
\def\@@GNUPLOTresult{\filterfromvalue{@@GNUPLOT-\@@GNUPLOTterminal}54}
\def\@@GNUPLOTfinalresult{\filterfromvalue{@@GNUPLOT-\@@GNUPLOTterminal}55}

%                    name       "set term" suffix conversion (system command)                         gnuplot's result          final result
% (suffix is probably not needed any more since full names were introduced)
\defineGNUPLOThandle
	{postscript}{postscript color}{ps}{texmfstart pstopdf --method=raw \GNUPLOTfile-ps.ps}
	{\GNUPLOTfile-ps.ps}{\GNUPLOTfile-ps.pdf}
\defineGNUPLOThandle
	{ps}{postscript color}{ps}{texmfstart pstopdf -dAutoRotatePages=/PageByPage --method=raw \GNUPLOTfile-ps.ps}
	{\GNUPLOTfile-ps.ps}{\GNUPLOTfile-ps.pdf}
\defineGNUPLOThandle
	{eps}{postscript color eps}{ps}{texmfstart pstopdf -dEPSCrop \GNUPLOTfile-eps.eps}
	{\GNUPLOTfile-eps.eps}{\GNUPLOTfile-eps.pdf}
\defineGNUPLOThandle
	{pdf}{pdf}{pdf}{}
	{\GNUPLOTfile-pdf.pdf}{\GNUPLOTfile-pdf.pdf}
\defineGNUPLOThandle
	{metapost}{mp}{mp}{texmfstart mptopdf \GNUPLOTfile-mp.mp}
	{\GNUPLOTfile-mp.mp}{\GNUPLOTfile-mp-0.pdf}
\defineGNUPLOThandle
	{mp}{mp}{mp}{texmfstart mptopdf \GNUPLOTfile-mp.mp}
	{\GNUPLOTfile-mp.mp}{\GNUPLOTfile-mp-0.pdf}
\defineGNUPLOThandle
	{png}{png}{png}{}
	{\GNUPLOTfile-png.png}{\GNUPLOTfile-png.png}
\defineGNUPLOThandle
	{context}{context textscale \@@GNUPLOT@term@context@textscale}{tex}{}
	{\GNUPLOTfile-ctx.tex}{\GNUPLOTfile-ctx.tex}
% \doifnotempty{\@@GNUPLOT@term@context@textscale}{textscale \@@GNUPLOT@term@context@textscale}

\def\defineGNUPLOTterminal
	{\dodoubleargument\dodefineGNUPLOTterminal}

\def\dodefineGNUPLOTterminal[#1][#2]%
	{\doifassignmentelse{#2}%
	 	% define a proper terminal
	 	{}% TODO
	 	% only define a synonym for that terminal
	 	{}% TODO
	}

%D {\sl terminal}: argument to be passed to gnuplot in the form of "set term {\sl terminal}"
%D {\sl defaultoptions}: options to be passed to gnuplot after terminal name:
%D      set term {\sl teminal} {\sl defaultoptions}
%D      not to be touched by users
%D {\sl suffix}: filename suffix
%D {\sl result}: the file to which gnuplot should output the result
%D      set output {\sl result}
%D {\sl convertwith}: command for conversion from gnuplot-generated file to a file that can be read by \CONTEXT; may be empty
%D {\sl finalresult}: file resulting after the conversion to be read by \CONTEXT

%D \POSTSCRIPT\ terminal
\defineGNUPLOTterminal
	[postscript]
	[terminal=postscript,
	 defaultoptions=color, % TODO: only if \setupcolors[state=start]
	 suffix=ps,
	 result=\GNUPLOTfile-ps.ps,
	 % TODO: -dAutoRotatePages=/PageByPage or remove that option from the default ones
	 convertwith={texmfstart pstopdf --method=raw \GNUPLOTfile-ps.ps},
	 finalresult=\GNUPLOTfile-ps.pdf,
	]
%D Define \type{ps} as a synonym for \type{postscript} terminal
\defineGNUPLOTterminal
	[ps]
	[postscript]

%D \EPS\ terminal:
%D - same terminal as for \POSTSCRIPT, but slightly different handling
\defineGNUPLOTterminal
	[eps]
	[terminal=postscript,
	 defaultoptions=eps color,
	 suffix=eps,
	 result=\GNUPLOTfile-eps.eps,
	 % TODO: -dEPSCrop
	 convertwith={texmfstart pstopdf -dEPSCrop \GNUPLOTfile-eps.eps},
	 finalresult=\GNUPLOTfile-eps.pdf,
	]

%D \PDF\ terminal
\defineGNUPLOTterminal
	[pdf]
	[terminal=pdf,
	 defaultoptions=,
	 suffix=pdf,
	 result=\GNUPLOTfile-pdf.pdf,
	 convertwith=,
	 finalresult=\GNUPLOTfile-pdf.pdf,
	]

%D \METAPOST\ terminal:
%D - deprecated: use the \CONTEXT\ terminal instead,
%D   which was built starting from the \METAPOST\ one,
%D   but improved in many aspects
\defineGNUPLOTterminal
	[mp]
	[terminal=mp,
	 defaultoptions=,
	 suffix=mp,
	 result=\GNUPLOTfile-mp.mp,
	 convertwith={texmfstart mptopdf \GNUPLOTfile-mp.mp},
	 % TODO: support for multiple graphics
	 % one would need a switch in mptopdf, so that a single pdf would be created instead of multiple ones
	 finalresult=\GNUPLOTfile-mp-0.pdf,
	]

\defineGNUPLOTterminal
	[metapost]
	[mp]

%D png terminal:
%D - bitmap
%D - no conversion needed
%D - new version pretty advanced
\defineGNUPLOTterminal
	[png]
	[terminal=png,
	 defaultoptions=,
	 suffix=png,
	 result=\GNUPLOTfile-png.png,
	 convertwith=,
	 finalresult=\GNUPLOTfile-png.png,
	]

%D \CONTEXT\ terminal (native)
\defineGNUPLOTterminal
	[context]
	[terminal=context,
	 defaultoptions=,
	 suffix=tex,
	 result=\GNUPLOTfile-ctx.tex,
	 convertwith=,
	 finalresult=\GNUPLOTfile-ctx.tex,
	]

\def\setupGNUPLOTterminal
	{\dodoubleargument\dosetupGNUPLOTterminal}

\def\dosetupGNUPLOTterminal[#1][#2]%
	{% TODO
		\getparameters[@@GNUPLOT@term@#1@][#2]
		% TODO: width & height
		% linejoin, linecap - I have to improve this !!!
	%	\edef\gp:term:context:linejoin{\@@GNUPLOT@term@context@linejoin}
	%	\edef\gp:term:context:linecap{\@@GNUPLOT@term@context@linecap}
		% dashed or solid lines?
		\doifsamestringelse{\@@GNUPLOT@term@context@dashed}{yes}% yes or no
			{\edef\@@GNUPLOT@term@context@is@dashed{true}}%
			{\edef\@@GNUPLOT@term@context@is@dashed{false}}%
	%		{\edef\gp:term:context:dashed{true}}%
	%		{\edef\gp:term:context:dashed{false}}%
	%	% dashlength scale
	%	\edef\gp:term:context:dashlength{\@@GNUPLOT@term@context@dashlength}
	%	% linewidth scale
	%	\edef\gp:term:context:linewidth{\@@GNUPLOT@term@context@linewidth}
	%	\doifsamestringelse{\@@GNUPLOT@term@context@points}{metapost}% tex or metapost
	}

\setupGNUPLOTterminal
	[context]
	[width=default,    % *default* (5in) | <dimension>
	 height=default,   % *default* (3in) | <dimension>
	 linejoin=rounded, % mitered | *rounded* | beveled
	 linecap=butt,     % *butt* | rounded (in gnuplot: round) | squared
	 dashed=yes,       % *yes* | no
	 dashlength=1,     % scaling factor for dash lengths
	 linewidth=1,      % scaling factor for line widths (1.0 means 0.4bp)
	 textscale=1,      % scaling factor for text labels
	 points=metapost]  % *metapost* | tex (Should points be drawn with MetaPost or TeX?)

% TODO: better scaling
\defineconversion
	[gnuplot:pointset]
	[{\scale[scale=800]{\mathematics{+}}},
	 {\scale[scale=800]{\mathematics{\times}}},
	 \mathematics{\ast},
	 {\scale[scale=700]{\mathematics{\square}}},
	 {\scale[scale=700]{\mathematics{\blacksquare}}},
	 \mathematics{\circ},
	 \mathematics{\bullet},
	 {\scale[scale=900]{\mathematics{\triangleup}}},
	 {\scale[scale=900]{\mathematics{\blacktriangle}}},
	 {\scale[scale=900]{\mathematics{\triangledown}}},
	 {\scale[scale=900]{\mathematics{\blacktriangledown}}},
	 {\scale[scale=800]{\mathematics{\lozenge}}},
	 {\scale[scale=800]{\mathematics{\blacklozenge}}}%,
%  {\rotate[rotation=45]{\mathematics{\square}}},
%  {\rotate[rotation=45]{\mathematics{\blacksquare}}},
	]

%\defineGNUPLOTcolor[red]    [r=1]
%\defineGNUPLOTcolor[green]  [g=1]
%\defineGNUPLOTcolor[blue]   [b=1]
%\defineGNUPLOTcolor[magenta][r=1,b=1]
%\defineGNUPLOTcolor[cyan]   [g=1,b=1]
%\defineGNUPLOTcolor[yellow] [r=1,g=1]
%\defineGNUPLOTcolor[black]  [s=0]
%\defineGNUPLOTcolor[orange] [r=1,g=.3,b=0]
%\defineGNUPLOTcolor[gray50] [s=.5]

%\defineGNUPLOTcolors
%	[default] % from PostScript
%	[red,green,blue,magenta,cyan,yellow,black,orange,gray50]

% TODO: testset is here only temporary & for testing
\setupGNUPLOT[program=gnuplot,\c!terminal=context,\c!purge=\v!yes,\c!options=,\c!pointset=gnuplot:pointset] % colors=postscript

%D Some additional typescripts which enable us using font "Arial" and "Helvetica"

%D Pretend the font to be serif as well, so that no "ss" switch is needed
\starttypescript [serif] [helvetica] [name]
  \definefontsynonym [Serif]            [Helvetica]
  \definefontsynonym [SerifBold]        [Helvetica-Bold]
  \definefontsynonym [SerifItalic]      [Helvetica-Oblique]
  \definefontsynonym [SerifSlanted]     [Helvetica-Oblique]
  \definefontsynonym [SerifBoldItalic]  [Helvetica-BoldOblique]
  \definefontsynonym [SerifBoldSlanted] [Helvetica-BoldOblique]
  \definefontsynonym [SerifCaps]        [Helvetica]
\stoptypescript

\beginOLDTEX

  \starttypescript [gnuplot] [texnansi,ec]
    \definetypeface [Helvetica] [rm] [serif]  [helvetica] [default] [encoding=\typescripttwo]
    \definetypeface [Helvetica] [ss] [sans]   [helvetica] [default] [encoding=\typescripttwo]
    \definetypeface [Arial]     [rm] [serif]  [helvetica] [default] [encoding=\typescripttwo]
    \definetypeface [Arial]     [ss] [sans]   [helvetica] [default] [encoding=\typescripttwo]
  \stoptypescript
  \usetypescript[gnuplot][ec]

\endOLDTEX
\beginNEWTEX

  \starttypescript [gnuplot]
    \definetypeface [Helvetica] [rm] [serif]  [helvetica] [default]
    \definetypeface [Helvetica] [ss] [sans]   [helvetica] [default]
    \definetypeface [Arial]     [rm] [serif]  [helvetica] [default]
    \definetypeface [Arial]     [ss] [sans]   [helvetica] [default]
  \stoptypescript
  \usetypescript[gnuplot]

\endNEWTEX

\stopmodule

\protect \doifnotmode{demo}{\endinput}


\starttext

\startGNUPLOTscript[exp]
set key bottom
set format x "%.1f"
set format y "%.1f"
set style fill solid 0.25 noborder
plot [0:3] 2/sqrt(pi)*exp(-x**2) t '$\frac{2}{\sqrt{\pi}}e^{-x^2}$' with filledcurves x1 lt 3, erf(x) lc 3 lw 2
\stopGNUPLOTscript

\useGNUPLOTgraphic[exp][width=.7\textwidth]


\startGNUPLOTinclusions
    set title "trigonometry"
\stopGNUPLOTinclusions

\startGNUPLOTscript[sin]
    plot sin(x)
\stopGNUPLOTscript

\startGNUPLOTscript[cos]
    plot cos(x)
\stopGNUPLOTscript

\useGNUPLOTgraphic[sin] \endgraf
\useGNUPLOTgraphic[cos] \endgraf
\useGNUPLOTgraphic[cos] \endgraf

\setupGNUPLOT[terminal=ps]  \useGNUPLOTgraphic[sin][object=no,height=2cm] \blank
%setupGNUPLOT[terminal=pdf] \useGNUPLOTgraphic[sin][object=no,height=2cm] \blank
%setupGNUPLOT[terminal=png] \useGNUPLOTgraphic[sin][object=no,height=2cm] \blank
\setupGNUPLOT[terminal=mp]  \useGNUPLOTgraphic[sin][object=no,height=2cm,options=color] \blank

\stoptext

