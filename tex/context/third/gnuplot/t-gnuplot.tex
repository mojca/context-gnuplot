%D \module
%D   [       file=t-gnuplot,
%D        version=2006.08.23,
%D          title=\CONTEXT\ Extra Modules,
%D       subtitle=\GNUPLOT\ Inclusion,
%D         author={Hans Hagen, Taco Hoekwater, Mojca Miklavec, Tobias Burnus},
%D           date=\currentdate,
%D      copyright=\PRAGMA]
%C
%C This module is part of the \CONTEXT\ macro||package and is
%C therefore copyrighted by \PRAGMA. See mreadme.pdf for
%C details.

%D TODO (Optimisations):
%D
%D \startitemize
%D \item optimize the number of gnuplot runs (if possible, gnuplot should be run only once)
%D \item optimize the number of times for loading/converting an already used graphic
%D \item pstopdf is a bit slow
%D \stopitemize
%D
%D TODO (Missing functionality):
%D
%D - a lot ...

\unprotect

% this doesn't help
%\startvariables dutch                     english
%                german                    czech
%                italian                   romanian
%                french
%
%      terminal: terminal                  terminal
%                terminal                  terminal
%                terminal                  terminal
%                terminal
%\stopvariables

\def\c!terminal {terminal}
\def\c!options  {options}
\def\c!pointset {pointset}

% \MPextensions; if possible, they should be specific to \startGNUPLOTgraphic (unvisible to MPcode)
% (probably something like \appendtoks ... to\evetyGNUPLOTgraphic)
\startMPextensions
	if unknown context_gplot: input mp-gnuplot.mp ; fi;
	gp_num_points_with_tex := \gp:num:pointswithtex;
	% override the main text color
	gp_color_foreground := \MPcolor{currentcolor};
	gp_color_lt[-2] := gp_color_foreground;
\stopMPextensions

% (hopefully) temporary solution for handling the color in
% \sometxt{...} withcolor red
\chardef\TeXtextcolormode\zerocount
% we need a \strut in front of labels for better vertical centering
%
% Aditya, thanks a lot for requesting it!
% Hans, thanks a lot for implementing this!
\definetextext[gp]{\strut}
% TODO:
%    \sometxt[gp][ss,20pt]{abc}
% should become equivalent to
%    \sometxt{\switchtobodyfont[ss,20pt]\strut abc}

%D For the old method of put_text("some text") using textext:
%D not used/supported any more, but still here for testing purposes
%D (to be removed in the future)
\forceMPTEXcheck{put_text}

\newcounter\GNUPLOTnumber

%D \macros
%D   {startGNUPLOTinclusions}

\long\def\startGNUPLOTinclusions
  {\bgroup
   \obeylines
   \catcode`\%=\@@letter
   \dostartGNUPLOTinclusions}

\long\def\dostartGNUPLOTinclusions#1\stopGNUPLOTinclusions
  {\gdef\GNUPLOTinclusions{#1}%
   \egroup}

\let\GNUPLOTinclusions\empty

%D \macros
%D   {startGNUPLOTscript}
%D
%D \starttyping
%D \startGNUPLOTscript{some name}
%D   plot sin(x)
%D \stopGNUPLOTscript
%D \stoptyping

\long\def\startGNUPLOTscript#1%
  {\bgroup
   \obeylines
   \catcode`\%=\@@letter
   \dostartGNUPLOTscript{#1}}

% FIXME:
% Math labels still don't work properly
%    plot x t '$\frac{2}{\sqrt{\pi}}e^{-x^2}$'
% will fail for example.
%
% Taco suggested to put \detokenize in front of the argument,
% but then I run into problems with newlines.
% I've lost too many nerves with ^^M already.
% This item should be somewhere on top of priority list,
% but I'm no TeXacker.
%
\long\def\dostartGNUPLOTscript#1#2\stopGNUPLOTscript
  {\doglobal\increment\GNUPLOTnumber
   \letgvalue{gps:n:#1}\GNUPLOTnumber
   \setgvalue{gps:d:\GNUPLOTnumber}{#2}%
%   \setgvalue{gps:d:\GNUPLOTnumber}{\detokenize{#2}}%
   \egroup}

\newtoks\everyGNUPLOTscript

%D We need to preserve newline characters (\TeX\ doesn't distinguish them from spaces).
%D Finding the proper solution has costed quite some nerves: most solutions resulted in \type{^M}.

\appendtoks
    \obeylines
    % Hans has put {;\outputnewlinechar} here, but I don't understand what the ';' is doing
    \def\obeyedline{\outputnewlinechar}
\to \everyGNUPLOTscript

%D Content of GNUPLOTscript has to be handled literally:
%D backslash (\letterbackslash) and percent (\%) may not interfere with \TeX.

\appendtoks
    \let\%\letterpercent
    \def\ {\letterbackslash\outputnewlinechar}%
\to \everyGNUPLOTscript

%D \macros
%D   {useGNUPLOTgraphic}

% 5 ways of calling it:
% - \useGNUPLOTgraphic{name}
% - \useGNUPLOTgraphic[name]
% - \useGNUPLOTgraphic[name][1,2,5]
% - \useGNUPLOTgraphic[name][width=.8\textwidth]
% - \useGNUPLOTgraphic[name][1,2,5][width=.8\textwidth]

\def\useGNUPLOTgraphic
  {\dotripleempty\douseGNUPLOTgraphic}

\def\douseGNUPLOTgraphic[#1][#2][#3]%
  {\ifthirdargument
     % as in \useGNUPLOTgraphic[name][1,2,5][width=.8\textwidth]
     \dodouseGNUPLOTgraphic[#1][#2][#3]%
   \else\ifsecondargument
     \doifassignmentelse{#2}
       % as in \useGNUPLOTgraphic[name][width=.8\textwidth]
       {\dodouseGNUPLOTgraphic[#1][][#2]}
       % as in \useGNUPLOTgraphic[name][1,2,5]
       {\dodouseGNUPLOTgraphic[#1][#2][]}
   \else\iffirstargument
     % as in \useGNUPLOTgraphic[name]
     \dodouseGNUPLOTgraphic[#1][][]%
   \else
     % as in \useGNUPLOTgraphic{name}
     \redouseGNUPLOTgraphic[#1][#2][#3]%
   \fi\fi\fi}

\def\redouseGNUPLOTgraphic[#1][#2][#3]#4%
  {\dodouseGNUPLOTgraphic[#4][][]}

\def\dodouseGNUPLOTgraphic[#1][#2][#3]%
  {\bgroup
   \doifdefined{gps:n:#1}
     {\edef\GNUPLOTnumber{\getvalue{gps:n:#1}}%
      \edef\GNUPLOTfile  {\bufferprefix gnuplot-\GNUPLOTnumber}%
      % call to gnuplot and processing/converting the graphics only needed in the first ConTeXt run
      \doifmode{*\v!first}{% TODO: and ONLY if it has not been processed yet !!!
        \bgroup
        \the\everyGNUPLOTscript
        \immediate\openout\scratchwrite=\GNUPLOTfile.plt
        \immediate\write\scratchwrite{set terminal \@@GNUPLOToutput\space\@@GNUPLOToptions}%
        \doifnotempty
          {\GNUPLOTinclusions}{\immediate\write\scratchwrite{\GNUPLOTinclusions}}%
        \immediate\write\scratchwrite{set output "\@@GNUPLOTresult"}%
        \immediate\write\scratchwrite{\getvalue{gps:d:\GNUPLOTnumber}}%
        \immediate\closeout\scratchwrite
        \egroup
        % TODO: check the state of write18 and warn the user, otherwise just everyone will start complaining that the module doens't work
        \processGNUPLOTscript
        \convertGNUPLOTgraphic
      }
     % for ConTeXt terminal only - read the result
     % TODO: \processGNUPLOTfile has to be moved away from here because of two reasons:
     % - so that one can load a file resulting from a GNUPLOT run explicitely and then use that graphic
     % - the file should not be loaded more than once (for efficiency)
     \doif{\@@GNUPLOTterminal}{context}{\processGNUPLOTfile[#1][\@@GNUPLOTresult]}
      % "ctxtools --purge" should delete the gnuplot script and other intermediate files
      % (but it seems that they are deleted automatically already; preferred or not?)
%      \registertempfile{\GNUPLOTfile.plt}
%      \registertempfile{\@@GNUPLOTresult}
%      \registertempfile{\@@GNUPLOTfinalresult}
     \doifelse{\@@GNUPLOTterminal}{context}%
     % for ConTeXt terminal only
     {\doifelse{#2}{}%
        % if no explicit figure number was specified, include all the figures
        {\doloop
           {\doifMPgraphicelse{gpg:#1:\recurselevel}
           {\scale[#3]{\reuseMPgraphic{gpg:#1:\recurselevel}}}
           {\exitloop}}}
        % if numbers were specified, include the figures specified in the list only
        {\begingroup
           \def\useGNUPLOTgraphicN##1{\doifMPgraphicelse
               {gpg:#1:##1}
               {\scale[#3]{\reuseMPgraphic{gpg:#1:##1}}}
               % if graphic doesn't exist: draw a dummy frame instead and warn the user
               {\scale[#3]{\framed[frame=on,width=5in,height=3in,align={middle,lohi}]{GNUPLOT graphic #1, Nr. ##1 doesn't exist}}}}
           \processcommalist[#2]\useGNUPLOTgraphicN
         \endgroup}}
     % for all the other terminals
     % TODO: add more safety switches (if pages don't exist for example)
     {\doifelse{#2}{}
        % if no explicite figure number was specified, include all the pages
        % TODO: properly handle METAPOST & PNG (only works for (E)PS & PDF) !!!
        {\getfiguredimensions[\@@GNUPLOTresult]%
           \dorecurse{\noffigurepages}{\externalfigure[\@@GNUPLOTfinalresult][page=\recurselevel,#3]}}
        % if numbers were specified, include the pages specified in the list only
        {\begingroup
           \def\useGNUPLOTgraphicN##1{\externalfigure[\@@GNUPLOTfinalresult][page=##1,#3]}
           \processcommalist[#2]\useGNUPLOTgraphicN
         \endgroup}}}
   \egroup}

%D \macros
%D   {setupGNUPLOT}

\def\setupGNUPLOT
%  {\dodoubleempty\getparameters[@@GNUPLOT]}
  {\dosingleempty\dosetupGNUPLOT}

\def\dosetupGNUPLOT[#1]%
  {\getparameters[@@GNUPLOT][#1]%
   % define all the necessary points according to the option "pointset"
   %
   % XXX: no idea why this is needed, but otherwise it complains that @@GNUPLOTpointset is undefined
   \edef\currentGNUPLOTpointset{\@@GNUPLOTpointset}%
   \startTeXtexts
    \doloop{\doifelseconversionnumber{\@@GNUPLOTpointset}{\recurselevel}
       % +500 is a hack (hopefully the plot doesn't contain more than 500 labels)
       % otherwise the points would be overwritten by labels with another \TeXtext:
       % it might need a fix in ConTeXt core
       {\TeXtext{\numexpr\recurselevel+500\relax}{{\convertnumber{\currentGNUPLOTpointset}{\recurselevel}}}}
       {\exitloop}}
   \stopTeXtexts
   % \gp:num:pointswithtex is passed to metapost, so that it knows
   % how many points are defined and chooses the proper point
   % form a set of the defined ones
   %
   % for safety reasons define \gp:num:pointswithtex to be equal to 1 (it can only increase later), otherwise:
   % - (something mod 0) won't work
   % - if conversion is not defined, the number will retain its old value (not desirable)
   \edef\gp:num:pointswithtex{1}
   \doloop{\doifelseconversionnumber{\@@GNUPLOTpointset}{\recurselevel}
       {\edef\gp:num:pointswithtex{\recurselevel}}
       {\exitloop}}%
   % TODO:
   % - issue a warning if user wants to use points with TeX,
   %   but pointset= is undefined (if the first point doesn't exist)
   % - no idea what happens if conversion is infinite,
   %   so try to stop at some reasonable value (100?)
   % - current implementation redefines the points even if only terminal type
   %   has been set to some other value (which is a stupid approach, but I
   %   wanted to have a working version first and will consider efficiency later)
   % - it may be that the old points remain defined if conversion changes
   %   (perhaps they should be undefined again?)
  }

% Hans has written this piece of code, but:
% - "start" and "/MIN" caused problems
% - pgnuplot is not much more "portable" than gnuplot
%   best thing to do is to create a "gnuplot.bat" somewhere in PATH
%
%\def\processGNUPLOTscript
%  {\doifelse\operatingsystem{mswin}
%     {\executesystemcommand{start /MIN pgnuplot \GNUPLOTfile.plt}} % start is needed else gp fails
%     {\executesystemcommand{gnuplot \GNUPLOTfile.plt}}}
\def\processGNUPLOTscript
  {\executesystemcommand{gnuplot \GNUPLOTfile.plt}}

% TODO: check if write18 is enabled; if not, issue a command and warn the user that running the module might be pointless or that he/she has to run gnuplot on the produced files manually
\def\convertGNUPLOTgraphic
  {\doifsomething\@@GNUPLOThandle{\executesystemcommand{\@@GNUPLOThandle}}}

\def\defineGNUPLOThandle#1#2#3#4#5#6% name; output; suffix; conversion-method; gnuplot's result; final result
  {\setvalue{@@GNUPLOT-#1}{{#2}{#3}{#4}{#5}{#6}}}

\def\@@GNUPLOToutput{\filterfromvalue{@@GNUPLOT-\@@GNUPLOTterminal}51}
\def\@@GNUPLOTsuffix{\filterfromvalue{@@GNUPLOT-\@@GNUPLOTterminal}52}
\def\@@GNUPLOThandle{\filterfromvalue{@@GNUPLOT-\@@GNUPLOTterminal}53}
\def\@@GNUPLOTresult{\filterfromvalue{@@GNUPLOT-\@@GNUPLOTterminal}54}
\def\@@GNUPLOTfinalresult{\filterfromvalue{@@GNUPLOT-\@@GNUPLOTterminal}55}

%                    name       "set term" suffix conversion (system command)                         gnuplot's result          final result
% (suffix is probably not needed any more since full names were introduced)
\defineGNUPLOThandle
	{postscript}{postscript color}{ps}{texmfstart pstopdf --method=raw \GNUPLOTfile-ps.ps}
	{\GNUPLOTfile-ps.ps}{\GNUPLOTfile-ps.pdf}
\defineGNUPLOThandle
	{ps}{postscript color}{ps}{texmfstart pstopdf -dAutoRotatePages=/PageByPage --method=raw \GNUPLOTfile-ps.ps}
	{\GNUPLOTfile-ps.ps}{\GNUPLOTfile-ps.pdf}
\defineGNUPLOThandle
	{eps}{postscript color eps}{ps}{texmfstart pstopdf -dEPSCrop \GNUPLOTfile-eps.eps}
	{\GNUPLOTfile-eps.eps}{\GNUPLOTfile-eps.pdf}
\defineGNUPLOThandle
	{pdf}{pdf}{pdf}{}
	{\GNUPLOTfile-pdf.pdf}{\GNUPLOTfile-pdf.pdf}
\defineGNUPLOThandle
	{metapost}{mp}{mp}{texmfstart mptopdf \GNUPLOTfile-mp.mp}
	{\GNUPLOTfile-mp.mp}{\GNUPLOTfile-mp-0.pdf}
\defineGNUPLOThandle
	{mp}{mp}{mp}{texmfstart mptopdf \GNUPLOTfile-mp.mp}
	{\GNUPLOTfile-mp.mp}{\GNUPLOTfile-mp-0.pdf}
\defineGNUPLOThandle
	{png}{png}{png}{}
	{\GNUPLOTfile-png.png}{\GNUPLOTfile-png.png}
\defineGNUPLOThandle
	{context}{context}{tex}{}
	{\GNUPLOTfile-ctx.tex}{\GNUPLOTfile-ctx.tex}


%D \macros
%D   {processGNUPLOTfile}

%D It's needed to input a file resulting from a gnuplot run (with ConTeXt terminal).
%D It reads the file and "saves" the metapost graphics defined in that file,
%D so that they can be used with \useGNUPLOTgraphic (low level: \reuseMPgraphic) later
%D
% \processGNUPLOTfile[NAME][filename]
\def\processGNUPLOTfile
  {\dodoubleargument\doprocessGNUPLOTfile}

\def\doprocessGNUPLOTfile[#1][#2]%
  {%\edef\@@GNUPLOTgraphicname{#1}%
   % we might need some additional arguments later on, but for now the figure number should suffice
   \long\def\startGNUPLOTgraphic
      {\dosingleargument\dostartGNUPLOTgraphic}
   \long\def\dostartGNUPLOTgraphic[##1]##2\stopGNUPLOTgraphic
      {\startreusableMPgraphic{gpg:#1:##1}##2\stopreusableMPgraphic}
   % input the file
   \readlocfile{#2}{}{} % the third argument should be: if file not found
   % this file is known
   % TODO \letgvalue{gpf:#1}\plusone
  }

%D \macros
%D   {startGNUPLOTpage}
%D
%D Used for standalone \GNUPLOT\ figures. Instead of having to
%D \type{\useGNUPLOTgraphic}, a figure is inserted directly using
%D \type{\startMPpage ... }\type{\stopMPpage}.
%D
%D A high-level command is provided if some more advance features will
%D be needed in the future and to be able to ensure backward compatibility.

\def\startGNUPLOTpage#1\stopGNUPLOTpage
	{\startMPpage#1\stopMPpage}

% TODO: better scaling
\defineconversion
  [gnuplot:pointset]
  [{\scale[scale=800]{\mathematics{+}}},
   {\scale[scale=800]{\mathematics{\times}}},
   \mathematics{\ast},
   {\scale[scale=700]{\mathematics{\square}}},
   {\scale[scale=700]{\mathematics{\blacksquare}}},
   \mathematics{\circ},
   \mathematics{\bullet},
   {\scale[scale=900]{\mathematics{\triangleup}}},
   {\scale[scale=900]{\mathematics{\blacktriangle}}},
   {\scale[scale=900]{\mathematics{\triangledown}}},
   {\scale[scale=900]{\mathematics{\blacktriangledown}}},
   {\scale[scale=800]{\mathematics{\lozenge}}},
   {\scale[scale=800]{\mathematics{\blacklozenge}}}%,
%  {\rotate[rotation=45]{\mathematics{\square}}},
%  {\rotate[rotation=45]{\mathematics{\blacksquare}}},
  ]


% TODO: testset is here only temporary & for testing
\setupGNUPLOT[\c!terminal=context,\c!options=,\c!pointset=gnuplot:pointset]

\protect \doifnotmode{demo}{\endinput}


\starttext

\startGNUPLOTscript{exp}
set key bottom
set format x "%.1f"
set format y "%.1f"
set style fill solid 0.25 noborder
plot [0:3] 2/sqrt(pi)*exp(-x**2) t '$\frac{2}{\sqrt{\pi}}e^{-x^2}$' with filledcurves x1 lt 3, erf(x) lc 3 lw 2
\stopGNUPLOTscript

\useGNUPLOTgraphic[exp][width=.7\textwidth]


\startGNUPLOTinclusions
    set title "trigonometry"
\stopGNUPLOTinclusions

\startGNUPLOTscript{sin}
    plot sin(x)
\stopGNUPLOTscript

\startGNUPLOTscript{cos}
    plot cos(x)
\stopGNUPLOTscript

\useGNUPLOTgraphic{sin} \endgraf
\useGNUPLOTgraphic{cos} \endgraf
\useGNUPLOTgraphic[cos] \endgraf

\setupGNUPLOT[terminal=ps]  \useGNUPLOTgraphic[sin][object=no,height=2cm] \blank
%setupGNUPLOT[terminal=pdf] \useGNUPLOTgraphic[sin][object=no,height=2cm] \blank
%setupGNUPLOT[terminal=png] \useGNUPLOTgraphic[sin][object=no,height=2cm] \blank
\setupGNUPLOT[terminal=mp]  \useGNUPLOTgraphic[sin][object=no,height=2cm,options=color] \blank

\stoptext

